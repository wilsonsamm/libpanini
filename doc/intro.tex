\documentclass{article}
\usepackage{thumbpdf}

\usepackage{color}
\definecolor{rltred}{rgb}{0.75,0,0}
\definecolor{rltgreen}{rgb}{0,0.5,0}
\definecolor{rltblue}{rgb}{0,0,0.75}

\title{The Panini Project}
\author{Sam M W}
\date{\today}

\begin{document}\label{start}

\maketitle

\section{What is this?}

The Panini Project provides:
\begin{itemize}
	\item A method of describing the grammar of a language
	\item A way of matching these rules against an input text, and conversely, a
way to generate new texts.
	\item An ontology (a way of representing the world)
\end{itemize}
By providing these things, it becomes possible to write computer programs that
treat language in some way. Of course, many applications already exist that (try
to) process natural language, for example:
\begin{itemize}
	\item Databases that are interrogable in (a subset of) plain English.
	\item Role-playing games that are instructed in (a subset of) plain English.
	\item Machine translation applications that take some input in Czech, for
instance, and have it translated to Swahili?
	\item IME: take some Japanese input in latin letters and have it fill the
Japanese scripts in
\end{itemize}

There have been many ad-hoc solutions to these problems until now. The Panini
Project hopes to unify these solutions (there, is after all, only the one
problem --- it's the problem of parsing and generating natural language) and
hopefully we'll come up with a way to process language that can be used in a
wide variety of applications, but with the effort having only been made once.

\section{What's in a name?}
The Panini Project (two capital P's; the T has no capital, except where typology
norms dictate otherwise) is named after P\=a\d{n}ini, the Sanskrit grammarian
from ancient India. I suggest that the name of the project will usually be
spelled without the diacritics, since they are awkward to type on most
computers.

Originally, this project was called the Tranny Project. Among my friends at
university, this was the common abbreviation of the phrases "translation",
"translation classes" and the like. Originally, this project was intended as a
machine translation platform, and the name stuck, despite the generalisation of
the project's scope. 

Since those days, whenever I have mentioned this project in a public space, many
people have been taken aback, or downright offended, since this word is also a
shortening of the word "transgendered". It has been such a barrier to sensible
discussion that I eventually renamed the project. Beware, though, that the word
"Tranny", which has nothing to do with transgenderism, pops up here and there in
the source code. It is being phased out, but means exactly the same thing as
"Panini".

\section{Architecture}
The Panini Project consists of several parts.

There is a programming language called \emph{panini}. It is Turing complete, but
is particularly geared toward describing language. Depending on how you like to
think, it can be seen as a beefed-up kind of Backus-Naur notation, or as a
procedural language that forks with every single function call. There should be
another PDF in this folder called programming.pdf which describes it in further
detail.

There is a compiler that compiles any panini source code into a lower-level
language. This second language is in turn interpreted at runtime. In this way,
it works similarly to the Java programming language because it too is compiled
into a bytecode, which is later interpreted.

There is a run-time library (these link to a computer program when it is being
built) that provides routines for using this language in your own project, and
the mechanisms to parse or generate a sentence in a given language, to
interrogate the meaning etc. In the demos/ folder, you may find various programs
that show how this run-time library is to be used.

\section{Compile-time}
This section describes what happens when you build the Panini Project and
install it onto your computer. The enclosed script ./build.sh takes care of all
this.

First up is to compile libpanini.a. This is the run-time library for
processing installed Panini files and samples of language.

Next, we compile tc, a compiler. This thing takes panini source code and
compiles it into a single file that can be read in and processed by libpanini.a.

The panini source code is partially written by hand, and this is mostly found in
the languages/ directory, under each language. These describe the language. To
know how these work, read programming.pdf.

There are also so-called "importers" in the imports/ directory. These are short
programs that take some input and uses it to generate panini source code. At the
time of writing, there are two such things:

\begin{itemize}
\item \emph{ekan}: Ekan is a program that takes the excellent and freely
available edict and kanjidic by Jim Breen, and converts the dictionary
definitions such that they may be used in the Panini Project.
\item \emph{pp}: pp, short for principles \& parameters, learns a language,
including its syntax and grammar, from the examples in a text file.
\end{itemize}


\label{end}\end{document}
