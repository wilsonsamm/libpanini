\documentclass{article}
\usepackage{thumbpdf}

\usepackage{listings}
\title{Panini, the programming language}
\author{Sam M W}
\date{\today}

\begin{document}\label{start}

\maketitle

\section{Introduction}

This book introduces and explains how a grammar and dictionary is programmed into Panini.

\section{Brief introduction to the Backus-Naur form}
The Backus-Naur Form describes a reduction of a long phrase into shorter ones. In short, that is what is known as a context-free 
grammar. Here is a possible grammar that matches some simple sentences, described in Backus-Naur:

\begin{lstlisting}
         sentence ::= <subject> <predication> <punct>
      predication ::= <verb-intransitive>
      predication ::= <verb-transitive> <object>
  verb-transitive ::= "love" | "see" | "dig" | "speak"
verb-intransitive ::= "love" | "weep" 
          subject ::= "I" | "you"
           object ::= "me" | "you" | "him" | "Norwegian"
            punct ::= "." | "?" | "!"
\end{lstlisting}
Note a few things: anything on the left of the ::= symbol is a \emph{non-terminal}; that is, a thing that must somehow be reduced 
down to a simpler things. Anything to the right of the ::= symbol is the thing or combination of things that the matching 
non-terminal may be matched to. These may include both terminals and non-terminals. A \emph{terminal} is something that does not
need further decomposition. Also note that a non-terminal non-terminal may decompose in several ways.

Here is how this may describe a sentence. We will look up "sentence" above, and see that it consists of a subject and a predication.
Looking up "subject", we see that it is one of "I" and "you". Let's choose "I". The next thing in the decomposition of a sentence is
the predication. We see that that it may either be a transitive verb plus an object, or an intransitive verb. Again, we'll choose 
one of the options. We will look at a transitive verb and an object. Look up verb-transitive, and choose one: "speak". The next 
thing in the decomposition of a predication is the object. One object is "Norwegian". The last thing that the sentence decomposed to
is some punctuation mark. We'll choose ".". We have composed, or decomposed, the sentence "I speak Norwegian.", a perfectly valid 
English sentence.

The grammar given above will also generate these sentences:
\begin{itemize}
	\item I weep.
	\item I love.
	\item I love him.
	\item You love Norwegian?
	\item You see Norwegian?
	\item You speak him.
\end{itemize}

As you can see, the grammar above produced some good English sentence, some dubious ones and some terrible ones. The reason for this
is that Backus-Naur describes a context-free grammar. Context is what says, for example, that Norwegian is spoken, but not seen. And
that we can see or love "him", but not speak "him".

\section{Brief introduction to the Panini}
Like Backus-Naur, Panini also describes grammar. But the difference is that Panini describes the contextual implications of a 
grammatical rule too. It also looks a bit different. Earlier, you saw a definition for "sentence" in Backus-Naur. Here is the same 
definition in Panini:
\begin{lstlisting}
(df sentence
	(constituent subject)
	(constituent predication)
	(constituent punct))
\end{lstlisting}
Note that everything is wrapped in parentheses. The first word in the parentheses is "df", which is short for "definition". The next
word is the symbol that we want to define, "sentence". After that come any number of instructions enveloped by parentheses. In this
example, there are instructions, each beginning with the word "constituent". These are instructions to try to parse a constituent 
named by the second word inside the parentheses.


\label{end}\end{document}
